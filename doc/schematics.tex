\documentclass[11pt]{standalone}
\usepackage{tikz}
\usepackage{fontenc}
	\usepackage{palatino}
\usepackage[hidelinks]{hyperref}
\usetikzlibrary{positioning}
\usepackage{circuitikz}
\usepackage{siunitx}
\begin{document}
\begin{circuitikz}[font={\bfseries\fontfamily{qag}\selectfont}]
  \def \xSepBoardToConnector {3}
  \def \yLED {12cm}
  \node[
  	rectangle, minimum width = 8 cm, minimum height = 19 cm,
  	anchor = north east, fill=cyan!10,draw=black
  	] at (-\xSepBoardToConnector,6) (board) {};
	  \node[below = of board.north, 	text width = 10cm, align=center,] 
	  		{\huge Nucleo-F103RB\\ \normalsize Development Board};
  \node[
  	rectangle, draw, 
	  minimum width = 3cm, minimum height = 3cm,
  ] (MCU) at (board.center)  {MCU};
  
  \def \myPushPull(#1){
  	\draw (0,0) 
  	to ++(1,0)
  	to[R=220~\si{\ohm}] ++(0,2)
   	to[C=10 nF] ++(0,2)
   	to 	++(1,0)
   	to [push button={#1}]++(0,-4)
   	to[short,-o] (0,0);
   	
   \draw  (1.5,4) to[short,*-] ++(0,0.5) node[above] {3.3~V};
   \draw  (1,0) to[short,*-, R=5 k$\Omega$] ++(0,-2) node[ground]{};
   }
	  \myPushPull(SeatSensor)
	  \draw[-triangle 60](0,0) -- node[pos = 0, above left] {to MCU PA6} 
	  	++(-\xSepBoardToConnector,0) 
	  	--++(-1cm,0) |-(MCU.30);
  \begin{scope}[yshift=-8cm]
  	\myPushPull(SeatSensor)
  	\draw[-triangle 60](0,0) -- node[pos = 0, above left] {to MCU PA7} 
  	++(-\xSepBoardToConnector,0) 
  	--++(-1cm,0) |-(MCU.east);
  \end{scope}
	\draw
		(-\xSepBoardToConnector,-\yLED) to[R={330~\si{\ohm}},f_={\mbox{PA4}}]
		++(2.5,0) to [led,a={warning indicator}] 
		++(4.3,0) node[ground]{};		
	\draw [-triangle 60]
		(MCU.south) |- (-\xSepBoardToConnector,-\yLED);
	
	\node[
		circle, minimum size=1cm, draw=black,fill=white,
		above left = 1cm and 1cm of MCU.north, 
		label={\small HSE CLK @ 8 MHz}
	](HSE Clock) {};
	\draw (HSE Clock.center) -- ++(2mm,0);
	\draw (HSE Clock.center) -- ++(0,3mm);
			
	\draw[-triangle 60] (HSE Clock.east) -| (MCU.north);
	
\end{circuitikz}

	
	
\end{document}

